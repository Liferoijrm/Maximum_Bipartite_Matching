\documentclass[12pt]{article}

\usepackage{sbc-template}
\usepackage[brazil]{babel}
\usepackage{graphicx}
\usepackage{url}
\usepackage{float}
\usepackage{listings}
\usepackage{color}
%\usepackage{todonotes}
\usepackage{algorithmic}
\usepackage{algorithm}
\usepackage{hyperref}
\usepackage{multirow}
\usepackage[table,xcdraw]{xcolor}
\hypersetup{
    colorlinks=true,
    linkcolor=black,
    filecolor=magenta,      
    urlcolor=blue,
    citecolor=black
    }
     
\sloppy

\title{Projeto 2\\ 
Problema de Alocação de Estudantes em Projetos (SPA)}

\author{
        Élvis Miranda, 24/1038700\\
        Gustavo Alves,  24/1020779\\
        Pedro Marcinoni, 24/1002396\\
}


\address{Dep. Ciência da Computação -- Universidade de Brasília (UnB)\\
  CIC0199 - Teoria e Aplicação de Grafos\\
  \today
  \email{elviscorreamirandajr@gmail.com, gusfring.a@gmail.com,  pedroextrer@gmail.com}
}

\begin{document}
\maketitle
\selectlanguage{american}
\begin{abstract}
    This paper presents the implementation of a solution for the Student-Project Allocation problem (SPA), applied to a scenario involving 200 students and 50 research projects, modeled through Graph Theory. The main objective was to develop a maximum stable matching algorithm that adheres to project capacity constraints (minimum and maximum quotas) and student qualification requirements (grades from 3 to 5), ensuring an impersonal and competitive selection process. Furthermore, two variations of the algorithm (student-proposing vs. project-proposing) were implemented and compared regarding their impact on stability and satisfaction.
\end{abstract}
\selectlanguage{brazil}
\begin{resumo}
    Este trabalho apresenta a implementação de uma solução para o Problema de Alocação de Estudantes em Projetos (SPA – Student-Project Allocation problem), aplicado a um cenário com 200 alunos e 50 projetos de pesquisa, modelado através da Teoria dos Grafos. O objetivo principal foi desenvolver um algoritmo de emparelhamento estável máximo que respeitasse as restrições de capacidade (mínima e máxima) dos projetos e os requisitos de qualificação dos alunos (notas de 3 a 5), garantindo uma seleção impessoal e competitiva. Além disso, duas variações do algoritmo (proposta pelo estudante vs. proposta pelo projeto) foram implementadas e comparadas em relação ao seu impacto na estabilidade e satisfação.
\end{resumo}

\section{Introdução}
\label{sec:Introducao}

\subsection{Objetivos}
\label{sec:Objetivos}
O objetivo deste projeto é projetar e implementar uma solução algorítmica para o Problema de Alocação de Estudantes em Projetos (SPA), modelando o cenário como um grafo bipartido. O trabalho visa garantir a alocação estável máxima entre 200 alunos e 50 projetos, respeitando restrições de capacidade e requisitos de qualificação. O programa deve ser capaz de:
\begin{itemize}
    \item Representar o cenário de alocação através de um grafo bipartido, onde um conjunto de vértices representa os alunos e o outro os projetos ofertados;
    \item Implementar variações do algoritmo de Gale-Shapley, conforme proposto por Abraham, Irving & Manlove (2007), para encontrar um emparelhamento estável máximo;
    \item Simular e visualizar a evolução do algoritmo, exibindo iterações do processo de propostas, emparelhamentos temporários e rejeições;
    \item Calcular e exibir métricas de satisfação, especificamente comparando a ordem de preferência dos alunos com o rank de classificação atribuído pelos projetos;
    \item Gerar uma matriz final de emparelhamento que evidencie o "ganho" ou "perda" de cada parte envolvida na alocação.
\end{itemize}

\subsection{Métodos}
\label{sec:Métodos}
Os métodos utilizados neste projeto envolvem a modelagem de grafos bipartidos e a aplicação de algoritmos de correspondência estável (stable matching) para a resolução de problemas de alocação com restrições. Para isso, o programa desenvolvido em \textbf{Python} realiza as seguintes etapas:

\begin{itemize}
    \item Processamento de Dados: Leitura e estruturação do arquivo de entrada, contendo as capacidades dos projetos (vagas mínimas e máximas), requisitos de notas e as listas de preferências dos alunos;
    \item Aplicação Algorítmica: Execução da lógica de emparelhamento baseada em propostas sucessivas, onde alunos aplicam para projetos e estes aceitam provisoriamente ou rejeitam com base em capacidade e qualificação (notas 3, 4 ou 5);
    \item Visualização Dinâmica: Renderização gráfica do grafo bipartido para monitorar a evolução das iterações e a comparação visual entre os resultados finais das duas variações propostas, utilizando cores distintas para diferenciar o estado das arestas (proposta ativa, emparelhamento temporário e rejeição);
    \item Análise de Métricas: Cálculo de índices de preferência e geração de uma matriz de resultados que cruza o rank do aluno na lista do projeto com o rank do projeto na lista do aluno, permitindo a validação da estabilidade e qualidade da solução encontrada.
\end{itemize}

\section{Procedimentos e Resultados}
\label{sec:Procedimentos}

\subsection{Modelagem e Estrutura de Dados}
\label{sec:Modelagem}

O problema foi modelado utilizando Programação Orientada a Objetos em Python. Foram instanciadas duas classes principais: \textit{Project} e \textit{Student}. A leitura dos dados (\texttt{entradaProj2.25TAG.txt}) gerou 50 objetos de projeto e 200 objetos de aluno.

Para garantir a integridade da simulação, as listas de preferência dos projetos foram pré-processadas, ordenando os candidatos aptos primeiramente por nota (decrescente) e, secundariamente, por ordem de chegada (ID).

\subsection{Algoritmos de Emparelhamento Implementados}
\label{sec:Algoritmos}

Foram desenvolvidas e comparadas duas variações do algoritmo SPA (\textit{Student-Project Allocation}), ambas visando a estabilidade, mas com lógicas de proposição opostas:

\subsubsection{Variação 1: SPA Orientado ao Estudante (Student-Proposing)}
Esta variação prioriza a escolha ativa dos alunos. O algoritmo itera enquanto houver alunos livres que ainda não propuseram para todas as opções de sua lista.
\begin{itemize}
    \item \textbf{Mecanismo de Proposta:} O aluno propõe para o seu projeto preferido disponível.
    \item \textbf{Critério de Aceitação e Substituição ("Mogging"):} Se o projeto tem vagas, o aluno entra. Se o projeto está cheio, ocorre uma competição com o pior candidato atual. O novo aluno substitui o atual se:
    \begin{enumerate}
        \item Sua nota for estritamente maior; OU
        \item As notas forem iguais, mas o novo aluno tiver listado aquele projeto em uma posição prioritária mais alta (menor índice de preferência) do que o aluno atual.
    \end{enumerate}
    \item \textbf{Objetivo:} Tende a produzir o emparelhamento estável "ótimo para o estudante".
\end{itemize}

\subsubsection{Variação 2: SPA Orientado ao Projeto (Project-Proposing)}
Nesta variação, os projetos realizam as propostas ativas para os alunos em suas listas de preferência.
\begin{itemize}
    \item \textbf{Mecanismo de Proposta:} Projetos com vagas ociosas propõem para o próximo aluno qualificado de sua lista.
    \item \textbf{Decisão do Aluno:}
    \begin{enumerate}
        \item Se o aluno estiver livre, aceita provisoriamente.
        \item Se o aluno já estiver alocado, ele compara a nova proposta com a atual. Se o novo projeto for "melhor" (estiver acima na sua lista pessoal de preferências), ele troca e libera a vaga no projeto anterior.
    \end{enumerate}
    \item \textbf{Objetivo:} Tende a produzir o emparelhamento estável "pessimal para o estudante" (ou ótimo para o projeto).
\end{itemize}

\subsection{Resultados Visuais e Comparativos}
\label{sec:Resultados}

A visualização gráfica permite observar a evolução das propostas e a saturação das vagas.

\begin{figure}[H]
    \centering
    \begin{minipage}{0.48\textwidth}
        \centering
        \includegraphics[width=\textwidth]{spa1.png}
        \caption{Grafo Final: Variação Orientada ao Estudante.}
        \label{fig:var1}
    \end{minipage}\hfill
    \begin{minipage}{0.48\textwidth}
        \centering
        \includegraphics[width=\textwidth]{spa2.png}
        \caption{Grafo Final: Variação Orientada ao Projeto.}
        \label{fig:var2}
    \end{minipage}
\end{figure}

\section{Conclusões}
\label{sec:Conclusao}

A implementação das duas variações do algoritmo SPA evidenciou o impacto de quem detém a iniciativa da proposta na satisfação final dos agentes.

Na \textbf{Variação 1 (Orientada ao Estudante)}, observou-se que os alunos tendem a conseguir vagas mais próximas ao topo de suas listas de preferências. O critério de desempate implementado (priorizando quem "quer mais" o projeto em caso de empate de notas) serviu como um fator decisivo para maximizar a utilidade percebida pelos discentes.

Na \textbf{Variação 2 (Orientada ao Projeto)}, a dinâmica favoreceu os projetos, que conseguiram preencher suas vagas seguindo rigorosamente sua ordem de classificação de notas. Embora estável (sem pares bloqueadores), essa solução tende a alocar os alunos em opções menos prioritárias de suas listas pessoais quando comparada à primeira variação.

Essa dualidade confirma a teoria de Gale-Shapley: embora a estabilidade seja garantida em ambos os casos, o lado que propõe (\textit{proposing side}) geralmente obtém o melhor resultado possível dentro do conjunto de soluções estáveis.

\label{sec:referencias}
\bibliographystyle{sbc}
\nocite{*} 
\bibliography{relatorio} 

\end{document}